\begin{comment}

\chapter{Autre partie}

Dans cette partie nous cherchons à décrire dans un premier temps [...], puis, c[...].

\section{Partie 1}

Intro

\subsection{Sous-partie 1}

\begin{figure}[!ht]
\begin{center}
\includegraphics[height=12cm]{autre_partie/image1}
\end{center}
\caption[autre partie générale]{autre partie image 1\protect\footnotemark}
%\floatfoot{Source: (Citation command)}
% avec le package "floatrow"
\end{figure}

%footnote protected pour apparaitre dans la légende d'une image
\footnotetext{Schéma d'après : \textit{Auteur 1 \& Propriétaire image}, LICENCE (cf. ref. \cite{cite4})}

\newpage{}

\subsection{Sous-partie 2}

\begin{figure}[!ht]
\begin{center}
\includegraphics[height=12cm]{autre_partie/image2}
\end{center}
\caption[autre partie]{autre partie globale de notre quelque chose}
\end{figure}

Nous retrouvons ici, blabla\footnote{Application bla - Interface blabla} [...].

\subsubsection{Sous-sous-partie 1}

Le bla (cf. ref. \cite{cite6}) est [...]:

\begin{itemize}
\item item1;
\item item2;
\item item3;
\item item4;
\item item5.
\end{itemize}

\newpage

\subsubsection{Sous-sous-partie 2}

%Les lignes :
% \setcounter{secnumdepth}{4}
% \setcounter{tocdepth}{4}
%dans le fichier "main.tex" permettent de faire en sorte que les paragraphes soient interprété comme des titres de niveau 5
\paragraph{Paragraphe 1 (agissant comme titre niveau 5)}
%forcer un saut de ligne
~\\
\hskip7mm

\begin{figure}[!ht]
\begin{center}
\includegraphics[height=6cm]{autre_partie/image3}
\end{center}
\caption[Structure d'unz autre chose]{Structure d'une autre chose\protect\footnotemark}
\end{figure}

Ce schéma représente bla.

\footnotetext{Schéma et explication d'après le wiki bla (cf. ref. \cite{cite0})}

\paragraph{Paragraphe 2}
~\\
\hskip7mm

%fixer les floats précédemment définis
%\FloatBarrier

Bla

\subparagraph{Sous-paragraphe 1}
~\\
\hskip7mm

Bla

\begin{figure}[H]
\begin{center}
\includegraphics[height=10cm]{autre_partie/image4}
\end{center}
\caption{Diagramme de truc}
\end{figure}

\subparagraph{Sous-paragraphe 2}
~\\
\hskip7mm

Bla\\

Bla

\subparagraph{Sous-paragraphe 3}
~\\
\hskip7mm

Bla

\subsubsection{Sous-sous-partie 3}

Bla

\section{Partie 2}

Bla

\footnotetext{D'après le schéma disponible sur la documentfation officielle disponible sur le site blalbla}

Bla

\subsection{Sous-partie 1}

Bla

\subsection{Sous-partie 2}

Bla

\paragraph*{Paragraphe 1 (n'apparaitra pas dans l'index)}
Bla

\paragraph*{Paragraphe 2}
Bla

\paragraph*{Paragraphe 3}
Bla

\subsection{Sous-partie 3}

Bla




\chapter{Analyse des besoins}

Intro

\section{Besoins fonctionnels}

Après une analyse des besoins fonctionnels du projet, nous avons défini deux sous catégories. D'un côté, les besoins [...], de l'autre, les besoins [...].

\subsection{Sous-partie 1}

Bla

\subsection{Sous-partie 2}

Bla

\newpage

\section{Besoins non-fonctionnels}

Comme précédemment, nous avons choisi de distinguer deux catégories pour les besoins non-fonctionnels. D'une part, nous avons les besoins non-fonctionnels pour les [...], et d'autre part ceux pour [...]. Nous avons aussi pris en compte les contraintes de développement, que nous détaillerons à la fin de cette partie.

\subsection{Sous-partie 1}

Bla\\

Aperçu du rendu souhaité :

\begin{figure}[!h]
\begin{center}
\includegraphics[height=10cm]{besoins/rendu}
\end{center}
\caption{Rendu attendu}
\end{figure}

\subsection{Sous-partie 2}

Bla

\newpage

\section{Développement}

Intro

\subsection{Tâches}

Bla\\

\section{Bilan récapitulatif}

Voici un tableau (cf. fig. 2.1) récapitulatif de notre analyse de l'existant...\\

%tableau centré à taille variable qui s'ajuste automatiquement suivant la longueur du contenu
\begin{figure}[!h]
\begin{center}
\begin{tabular}{|l|l|l|l|l|}
\hline
Solution & Critère 1 & Critère 2 & Critère 3 & Critère 4\\
\hline
Solution 1(cf. ref. \cite{cite0}) & Oui & Oui & Oui & Oui \\
Solution 2(cf. ref. \cite{cite1}) & Oui & Oui & Oui & Non \\
Solution 3(cf. ref. \cite{cite2}) & Oui (sauf telle chose) & Non & Non & Oui\\
Solution 4(cf. ref. \cite{cite3}) & Oui& Non & Oui & Non\\
Solution 5(cf. ref. \cite{cite4}) & Oui (uniquement ceux-ci) & Non & Oui & Non\\
\hline
\end{tabular}
\end{center}
\caption{Tableau récapitulatif des solutions}
\end{figure}


%tableau à taille fixée sur certaines colonnes (param sur la ligne \begin{tabularx}, voir wiki pour plus d'info sur la syntaxe
\begin{figure}[!h]
\begin{center}
\begin{tabularx}{17cm}{|c|p{6cm}|X|}
\hline
Priorité & Nom & Raison\\
\hline
1 & Tache 1 & Doit être vérifié en premier car sinon [...] \tabularnewline
2 & Tache 2 & On doit pouvoir [...] \tabularnewline
3 & Tache 3 & Comme les principales fonctionnalités permettant de tester sont opérationnelles, nous pouvons passer à cette tâche. \tabularnewline
4 & Tache 4 & Parce que [...] \tabularnewline
5 & Tache 5 & La tache 5 fait partie des principales [...]. \tabularnewline
6 & Tache 6 & Dernière fonctionnalité essentielle à mettre en place. \tabularnewline
7 & Tache 7 & Non-essentiel, mais apporterait un plus au projet. \tabularnewline
8 & Tache 8 & Non-essentiel, mais apporterait un plus au projet. \tabularnewline
\hline
\end{tabularx}
\end{center}
\caption{Tableau récapitulatif des tâches}
\end{figure}

\subsection{Tests}

Bla\\

\begin{figure}[!h]
\begin{center}
\begin{tabularx}{17cm}{|p{6cm}|X|}
\hline
Fonctionnalité & Test\\
\hline
Fonction 1 & Quand [...], vérifier [...]. \tabularnewline
& Et quand [...], vérifier [...]. \tabularnewline
Fonction 2 & Vérifier [...]. \tabularnewline
Fonction 3 & Vérifier [...]. \tabularnewline
Fonction 4 & Avoir [...]. \tabularnewline
Fonction 5 & Accéder à [...]. \tabularnewline
& Vérifier que [...]. \tabularnewline
Fonction 6 & Accéder à [...]. \tabularnewline
& Et vérifier [...]. \tabularnewline
Fonction 7 & Installer [...]. \tabularnewline
& Vérifier [...]. \tabularnewline
Fonction 8 & Compter [...]. \tabularnewline
\hline
\end{tabularx}
\end{center}
\caption{Tableau récapitulatif des tests}
\end{figure}

\end{comment}