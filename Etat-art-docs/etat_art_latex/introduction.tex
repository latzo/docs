\renewcommand{\abstractnamefont}{\normalfont\Large\bfseries}
%\renewcommand{\abstracttextfont}{\normalfont\Huge}
\renewcommand{\abstractname}{Introduction}

\begin{abstract}
\hskip7mm

\begin{spacing}{1.3}

Afin d’appliquer les notions enseignées en Cours de Systèmes Multi Agents, nous avons eu à réaliser une étude de cas avec GAMA,un environnement de développement orienté modélisation et simulation de systèmes à base d'agents. Cet étude de cas concerne une \textbf{\textit{exploitation d'un territoire minier par une population de robots}}. Les robots doivent trouver tous les minerais présents dans le territoire et les amener à une base.\\
L'étude de cas est décliné en deux versions : une première version avec des robots réactifs et une deuxième version avec des robots cognitifs. Nous avons structuré notre rapport de la façon suivante :
Tout d'abord, dans une première partie nous faisons une spécification générale dans laquelle nous fixons les définitions de chacun des éléments manipulés dans notre étude de cas. Puis dans une seconde partie, nous présentons la première version du robot, le robot réactif. Ensuite, dans une troisième partie, nous déclinerons le robot cognitif. Puis nous ferons des test de sensibilité et comparaisons des deux versions avant de finir par une conclusion.

\end{spacing}
\end{abstract}
